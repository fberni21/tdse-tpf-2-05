\documentclass{beamer}

\usetheme{metropolis}
\usepackage{xcolor}

\title{Sistema de Control \\ para Cámara de Termovacío}
\subtitle{Trabajo Práctico Final \\ Taller de Sistemas Embebidos}

\author{Franco Berni \,•\, Manuel Hirsch \,•\, Juan Ignacio Giachetti \,•\, Lautaro García Vitale}

\institute{Facultad de Ingeniería -- UBA}

\date{1\textsuperscript{er} Cuatrimestre 2025}

\begin{document}
	
	\frame{\titlepage}
	
\begin{frame}{¿Qué es una cámara de termovacío?}
	
	\small
	
	Una \textbf{cámara de termovacío} es un sistema cerrado que permite 
	\textbf{controlar simultáneamente la temperatura y la presión} 
	en un entorno aislado.
	
	
	\vspace{0.5cm}
	
	\textbf{Aplicaciones típicas}
	
	\begin{itemize}
		\item Ensayos de componentes electrónicos
		\item Testing aeroespacial
		\item Pruebas de materiales
	\end{itemize}
	
\end{frame}

\begin{frame}{Producto Mínimo Viable (MVP)}
	
	\small
	
	El sistema desarrollado corresponde a un \textbf{producto mínimo viable} que
	replica el comportamiento lógico de una cámara de termovacío real,
	utilizando elementos simples para simular sensores y actuadores.
	
	\vspace{0.5cm}
	
	\begin{tabular}{p{4.5cm} p{5cm}}
		\textbf{Sistema real} & \textbf{Implementación en el MVP} \\
		\hline
		Sensor de temperatura & Potenciómetro leído por ADC \\
		Sensor de presión & Potenciómetro leído por ADC \\
		Calentador / refrigeración & LED indicador \\
		Bomba de vacío & LED indicador \\
		Sistema de alarma & Buzzer \\
		Panel de control & LCD + botones / switches \\
	\end{tabular}
	
\end{frame}

\begin{frame}{Prototipo del sistema (MVP)}
	
	\centering
	
	\small
	Implementación física del MVP utilizando una placa STM32 y periféricos
	para simular sensores, actuadores e interfaz de usuario.
	
	\vspace{0.4cm}
	
	\includegraphics[width=0.5\textwidth]{img/mvp.jpeg}
	
\end{frame}

\begin{frame}{Statechart del sistema}
	
	\centering
	\includegraphics[width=0.85\textwidth]{img/sc_sys.drawio.pdf}
	
	\vspace{0.4cm}
	
	\small
	\textbf{Funcionamiento general}
	
	\begin{itemize}
		\item En modo \textbf{NORMAL}, si llega un evento de encendido, avisa a los sistemas de presión y temperatura que controlen.
		\item En modo \textbf{MENÚ} se configuran los parámetros.
		\item Estando en modo \textbf{NORMAL}, si algún valor excede los límites de alarma, se entra en modo \textbf{ALARMA}.
		\item Se navega con el botón de Enter.
	\end{itemize}
	\vspace{0.6cm}
	{\tiny Implementación: \texttt{code/app/src/task\_system.c}}
	
\end{frame}

\begin{frame}{Statechart del menú}
	
	\centering
	\includegraphics[width=0.85\textwidth]{img/sc_menu.drawio.pdf}
	
	\vspace{0.1cm}
	
	\small
	\textbf{Funcionamiento del menú}
	
	\begin{itemize}
		\item El usuario accede al \textbf{MENÚ} mediante los botones del sistema.
		\item Se pueden modificar los \textbf{setpoints de temperatura y presión} y las \textbf{bandas de histéresis}.
		\item Se puede configurar el uso de \textbf{alarmas}.
		\item Una vez confirmados los cambios, los parámetros se \textbf{guardan en memoria}.
	\end{itemize}
	
	{\tiny Implementación: \texttt{code/app/src/task\_menu.c}}
	
\end{frame}

\begin{frame}{Control de temperatura y presión}
	
	\begin{columns}
		
		\column{0.48\textwidth}
		\centering
		\textbf{Control de temperatura}
		
		
		\includegraphics[width=\linewidth]{img/sc_temp.drawio.pdf}
		

		
		{\tiny Implementación: \texttt{code/app/src/task\_temp.c}}
		
		\column{0.48\textwidth}
		\centering
		\textbf{Control de presión}
		
		
		\includegraphics[width=\linewidth]{img/sc_press.drawio.pdf}
		
		
		{\tiny Implementación: \texttt{code/app/src/task\_press.c}}
		
	\end{columns}
	
	\vspace{0.4cm}
	
	\small
	\begin{itemize}
		\item Ambos controles utilizan una lógica basada en \textbf{setpoints} y \textbf{bandas de histéresis}.
		\item El sistema monitorea continuamente los valores de temperatura y presión.
		\item Cuando una variable sale de la banda permitida, se activa el actuador correspondiente.
	\end{itemize}
	
\end{frame}

\begin{frame}{Ejemplo: línea de tiempo}
	
	\centering
	\includegraphics[width=1\textwidth]{img/linea_tiempo.drawio.pdf}
	
	
\end{frame}

\begin{frame}{Adquisición de señales analógicas}
	
	\small
	
	\begin{itemize}
		\item El sistema mide las variables de \textbf{temperatura} y \textbf{presión} mediante dos entradas analógicas del microcontrolador.
		
		\item Se simulan utilizando \textbf{potenciómetros}, cuyas tensiones son convertidas por el \textbf{ADC del STM32}.
		
		\item Se convierten ambos canales en cada ciclo
		
		\item Los resultados de las conversiones se transfieren automáticamente a memoria mediante \textbf{DMA (Direct Memory Access)}.
		
		\item El uso de DMA permite almacenar las mediciones en un \texttt{adc\_buffer} \textbf{sin intervención del CPU}, reduciendo la carga del procesador y permitiendo que se sigan ejecutando las tareas.
		
		\item \texttt{task\_adc} lee este buffer y copia los valores a la estructura compartida \texttt{shared\_data} para que el resto del sistema utilice las mediciones.
	\end{itemize}
	
	\vspace{0.2cm}
	{\tiny Implementación: \texttt{code/app/src/task\_adc.c}}
	
\end{frame}

\begin{frame}{Escritura no bloqueante en EEPROM}
	
	\small
	
	\textbf{Problema}
	
	\begin{itemize}
		\item Se busca que cada ciclo dure menos de \textbf{1 ms}.
		\item Una escritura en EEPROM tarda aproximadamente \textbf{5 ms}.
		\item Una implementación bloqueante detendría el sistema durante varias iteraciones.
	\end{itemize}
	
	\vspace{0.3cm}
	
	\textbf{Solución}
	
	\begin{itemize}
		\item Se utiliza \textbf{I2C con interrupciones} (\texttt{HAL\_I2C\_Mem\_Write\_IT}).
		\item La transmisión se realiza en segundo plano.
		\item Un \textbf{callback} indica cuándo finaliza la operación.
	\end{itemize}
	
	\vspace{0.2cm}
	{\tiny Implementación: \texttt{code/app/src/eeprom.c}}
	
\end{frame}

\begin{frame}{Actualización no bloqueante del display}
	
	\small
	
	\textbf{Problema}
	
	\begin{itemize}
		\item La comunicación con el display LCD se realiza por \textbf{I2C}.
		\item Escribir múltiples caracteres puede tomar mucho tiempo.
		\item Se tiene un problema similar al anterior.
	\end{itemize}
	
	\vspace{0.3cm}
	
	\textbf{Solución}
	
	\begin{itemize}
		\item Se implementó una \textbf{cola de comandos} para el display.
		\item Las distintas tareas envían comandos a la cola.
		\item La tarea de display procesa los comandos \textbf{gradualmente en cada iteración}.
	\end{itemize}
	
	\vspace{0.2cm}
	{\tiny Implementación: \texttt{code/app/src/task\_display.c}}
	
\end{frame}

	
\end{document}